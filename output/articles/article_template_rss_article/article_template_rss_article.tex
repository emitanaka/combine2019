
\documentclass{statsoc}
\usepackage{graphicx}
\usepackage{listings}
\usepackage{color}
\usepackage{amssymb, amsmath, geometry}
\usepackage{natbib}
\usepackage{hyperref}

\makeatletter
\def\maxwidth{\ifdim\Gin@nat@width>\linewidth\linewidth\else\Gin@nat@width\fi}
\def\maxheight{\ifdim\Gin@nat@height>\textheight\textheight\else\Gin@nat@height\fi}
\makeatother
% Scale images if necessary, so that they will not overflow the page
% margins by default, and it is still possible to overwrite the defaults
% using explicit options in \includegraphics[width, height, ...]{}
\setkeys{Gin}{width=\maxwidth,height=\maxheight,keepaspectratio}

\title[Untitled]{Untitled}

\author[Author 1 et. al.]{Author 1}
\address{Affiliation,
City,
Country}
\email{email1@example.com}

\author[Author 1 et. al.]{Author 2}
\address{Affiliation,
Los Angeles,
Country}
\email{email2@example.com}

% BIBLIOGRAPHY
\usepackage[authoryear]{natbib}
\bibpunct{(}{)}{;}{a}{}{,}


\begin{document}

\begin{abstract}
Abstract goes here
\end{abstract}
\keywords{keywords}

\hypertarget{introduction}{%
\section{Introduction}\label{introduction}}

This template demonstrates some of the basic latex you'll need to know
to create a RSS article.

\hypertarget{r-code}{%
\section{R code}\label{r-code}}

Can be inserted in regular R markdown blocks.

\begin{lstlisting}[language=
R
]
x <- 1:10
x
\end{lstlisting}

\begin{verbatim}
##  [1]  1  2  3  4  5  6  7  8  9 10
\end{verbatim}

\bibliographystyle{rss}
\bibliography{bibliography}
\end{document}
