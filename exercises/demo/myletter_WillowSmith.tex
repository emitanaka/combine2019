\documentclass[11pt,]{letter}
\usepackage{mathpazo}
\usepackage{amssymb,amsmath}
\usepackage{ifxetex,ifluatex}
\usepackage{fixltx2e} % provides \textsubscript
\ifnum 0\ifxetex 1\fi\ifluatex 1\fi=0 % if pdftex
  \usepackage[T1]{fontenc}
  \usepackage[utf8]{inputenc}
\else % if luatex or xelatex
  \ifxetex
    \usepackage{mathspec}
  \else
    \usepackage{fontspec}
  \fi
  \defaultfontfeatures{Ligatures=TeX,Scale=MatchLowercase}
\fi
% use upquote if available, for straight quotes in verbatim environments
\IfFileExists{upquote.sty}{\usepackage{upquote}}{}
% use microtype if available
\IfFileExists{microtype.sty}{%
\usepackage{microtype}
\UseMicrotypeSet[protrusion]{basicmath} % disable protrusion for tt fonts
}{}
\usepackage[margin=1in]{geometry}
\usepackage[unicode=true]{hyperref}
\hypersetup{
            pdfauthor={Emi Tanaka},
            pdfborder={0 0 0},
            breaklinks=true}
\urlstyle{same}  % don't use monospace font for urls
% Make links footnotes instead of hotlinks:
\renewcommand{\href}[2]{#2\footnote{\url{#1}}}
\IfFileExists{parskip.sty}{%
\usepackage{parskip}
}{% else
\setlength{\parindent}{0pt}
\setlength{\parskip}{6pt plus 2pt minus 1pt}
}
\setlength{\emergencystretch}{3em}  % prevent overfull lines
\providecommand{\tightlist}{%
  \setlength{\itemsep}{0pt}\setlength{\parskip}{0pt}}
\setcounter{secnumdepth}{0}
% Redefines (sub)paragraphs to behave more like sections
\ifx\paragraph\undefined\else
\let\oldparagraph\paragraph
\renewcommand{\paragraph}[1]{\oldparagraph{#1}\mbox{}}
\fi
\ifx\subparagraph\undefined\else
\let\oldsubparagraph\subparagraph
\renewcommand{\subparagraph}[1]{\oldsubparagraph{#1}\mbox{}}
\fi

% set default figure placement to htbp
\makeatletter
\def\fps@figure{htbp}
\makeatother

\usepackage{graphicx,grffile}
\signature{\includegraphics{signature.png}\\Emi Tanaka}

\date{4th October 2019}

\address{Menzies Building\\
Monash University}

\usepackage{mdframed} % color is loaded by mdframed
\definecolor{greyborder}{RGB}{221,221,221}
\definecolor{greytext}{RGB}{119,119,119}
\newmdenv[rightline=false,bottomline=false,topline=false,linewidth=3pt,linecolor=greyborder,skipabove=\parskip]{blockquote}
\renewenvironment{quote}{\begin{blockquote}\list{}{\rightmargin=0em\leftmargin=0em}%
\item\relax\color{greytext}\ignorespaces}{\unskip\unskip\endlist\end{blockquote}}


\begin{document}


\begin{letter}{Willow Smith\\
Carlsaw Room 353\\
The University of Sydney\\}
\opening{Dear Willow,}

Welcome to the ``Communicating with Data via R Markdown'' workshop. This
workshop is kindly sponsored by \href{https://combine.org.au/}{COMBINE}
and organised by \href{https://twitter.com/_mengbo}{Mengbo Li}.

This workshop teaches you how to use R Markdown but teaches little in
communicating effectively. The aim of this workshop is so that you can
get used to R Markdown, which integrates code and text together with
ease.

\begin{quote}
To effectively communicate, we must realize that we are all different in
the way we perceive the world and use this understanding as a guide to
our communication with others.
\end{quote}

For effective communication, seek to know your audience and extract that
story that the data are telling.

\begin{quote}
The two words `information' and `communication' are often used
interchangeably, but they signify quite different things. Information is
giving out; communication is getting through.
\end{quote}

Learning R Markdown is a step towards getting accurate information out
but communication requires much more. This is why there is a benefit in
coming to a workshop where you can interact with your peers.

\begin{quote}
Communication leads to community, that is, to understanding, intimacy
and mutual valuing.
\end{quote}

I hope you can take this opportunity to get to know each other.

\longindentation=0pt
\closing{Sincerely,}
\cc{Mengbo Li}

\end{letter}

\end{document}
